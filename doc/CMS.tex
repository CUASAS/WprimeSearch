

\subsection{The Large Hadron Collider}

The European Organization for Nuclear Research, CERN (by its French acronym)
established 66 years ago in 1964 runs the largest particle physics laboratory in
the world and it is home of dozens of experiments in a wide range of physics
across its large infraestructure located in the Swiss-French border. Experiments
in the whole length scale, from particle physics with collaborations such as
CMS and ATLAS, to astronomy and earth science with experiments such as CAST
and CLOUD.

The Large Hadron Collider (LHC) is located in the French-Swiss border and
extends through a perimeter of 17 miles (27 $km$). The LHC allow us to collide
protons and heavy ions (lead nuclei) together and study the products of these
collisions.

The LHC takes advantage on already existing infraestructure which is used in the
different stages of the acceleration process.

**mention the antyhydrogen atom at CCERN

A more detailed description of the LHC can be found ***somewhere.

The Compact Muon Solenid detector (CMS) is one of the four points across the LHC
in which hadronic collisions take place every 25ns, around hundreds of thousands
of millions of protons (or heavy ions) are injected from the beam pipe
into the center of the cylindrical detector. From this vast number of protons
only a handful collide, for instance, during the 2016 data taking period a mean
of 23 collisions per bunch crossing during 2016 was seen. \footnote{ This
  phenomenom is known as pileup. The fact that we are not looking at a single
  head on collision requires additional considerations that will be further
  explored later in this section.}

Once the hadrons collide, the products of each collision make its way through
the different subdetectors. The very first detector, the Tracker detector, is
meant to provide a way to keep track of the charged particles generated during
the collision, this tracks allow us to assign identities to the particles and
measure its momenta. A longer description of the tracker is given in section

The Electromagnetic Calorimeter (ECAL) allow us to measure the energy of electrons,
positrons and photons, and stop them in the process. The Hadronic Calorimeter (HCAL)
allow us to get measurements from hadronic-like particle, such as pions ***. And Finally
the Muon chambers *allow us to keep track of long-lived particles, the presence
of muons in a reaction is a sign of a head on, high energy collision, and therefore
of interesting physics. For instance, a 4 muon signature was found in events
containing the well known discovery of the Higgs.

The CMS detector and its subsystems have passed through different development
stages. The current analysis was

\subsection{CMS Coordinate system}

*** need to describe the coordinate system ****

\subsection{The Tracker Detector}

The tracker subdetector is in charge of keeping track of the trajectories of
charged particules produced in the collision (collectively referred as \em{tracks}),
these trajectories are reconstructed by combining the hits in the different layers
of silicon modules as the outgoing particle make its way through the detector.
Due to the presence of the homongeneus magnetic field, these tracks *allow us to
determine not only the charge (different electric charges will bend in opposite directions)
but also determine its transverse momentum $P_{T}$, the reconstruction of the primary
and secondary vertex relay on the accuracy of the measurements provided by the tracker.

It is composed by two categories based on the material that it uses: the pixel with
an spatial resolution of about 10 $\mum$ in $r-\phi$ and up to $40\mum$ in $z$
and the strips

A new pixel was installed at the end of 2016, which gave access to an improved
protection against a harsher environment (radiation hardness), an additional
layer of silicon modules in the barrel and the endcap, and an overall reduction
of the material budget as measured in hadronic interaction length. *** show some plots ***

There's an image showing the pixel differences TS2020_009_GOODCMS.pdf page 48

@techreport{Lipinski:2265423,
      author        = "Lipinski, Martin",
      title         = "{The Phase-1 Upgrade of the CMS Pixel Detector}",
      institution   = "CERN",
      collaboration = "CMS Collaboration",
      address       = "Geneva",
      number        = "CMS-CR-2017-135. 06",
      month         = "May",
      year          = "2017",
      reportNumber  = "CMS-CR-2017-135",
      url           = "https://cds.cern.ch/record/2265423",
      doi           = "10.1088/1748-0221/12/07/C07009",
}


Describe the basic physics mechanism that allows each subsystem to work.


\subsection{The ECAL}

The Electromagnetic Calorimeter is built on lead tungsten PbWO4 crystals.

\subsection{The HCal}

The HCAL was built on russian bullet shells


\subsection{The Muon Chambers}

The muon chamber and the CSC

\subsection{The trigger system}

The amount of data generated at such frequency in LHC collisions is a stress
test for even the most current technology available in the market. Due to
limitations in the electronics, data transfer speed and others the CMS detector
accounts for


*** Every bunch collision would require a million gigabytes (a petabyte) to be stored
*** a bunch crossing every 25ns (40 million per second) continuously for months
*** makes this unfeasible, nafordable, a.k.a impossible

The L1 trigger takes an input of 40MHz and output 100KHz, then the hundreds of
thousands of event per second are further filtered out trought the High Level
Trigger to take an output of 200Hz. So the number of discarded events are
around 99.9995%

L1 Trigger or online trigger

*** We can make a comment of the different formats AOD,NanoAOD in terms of weight/event


\subsection{Computing infraestructure}

* Talk about simulation?

* Simulation comprises multiple steps: Event generation, detector simulation, digitazation
reconstruction, etc.

* By default the vertex is (0,0,0) however, in real life this is not true that's why
we have a VertexSmearing process that introducess a distribution arount this default vertex

* sensitive detectors are the ones that has a readout

* Most of the simulations are the same, except for the generation step, that's the purpose
of cmsDriver, help us to write configuration files based on approved simulation standards

