\documentclass[10pt]{article}
\usepackage[utf8]{inputenc}
\usepackage{url}
\usepackage{hyperref}
\usepackage{amsmath}
\usepackage{amsfonts}
\usepackage{amssymb}
\usepackage{graphicx}
\usepackage{float}
\usepackage{lipsum}
\usepackage{multicol}
\usepackage{xcolor}
\usepackage[font=small]{caption}
%\addtolength{\abovecaptionskip}{-3mm}
%\addtolength{\textfloatsep}{-5mm}
\setlength\columnsep{20pt}

\usepackage[a4paper,left=1.50cm, right=1.50cm, top=1.50cm, bottom=1.50cm]{geometry}


\author{}

\title{SEARCH FOR RESONANT WZ PRODUCTION IN THE FULLY LEPTONIC FINAL STATE AT $13~TeV$ WITH THE CMS DETECTOR}

\begin{document}

\begin{center}
  {\Large \textbf{SEARCH FOR RESONANT WZ PRODUCTION IN THE FULLY LEPTONIC FINAL STATE AT $13~TeV$ WITH THE CMS DETECTOR}}\\
  \vspace{1em}
         {\large Andr\'es Vargas Hernandez}\\
         \vspace{1em}
%         \textit{The Catholic University of America}
\end{center}

\vspace{5mm}

%\begin{multicols*}{2}

\section{Statement of the problem}

The electromagnetic, weak, strong, and gravitational forces are known to be fundamental in the structure of the universe. With the exception of gravity, these forces are well described by the Standard model (SM), and its predictions have been successfully tested in numerous experiments with remarkable precision.

As suggested initially by Schwinger ~\cite{schwinger1957}, and later developed by Weinberg ~\cite{weinberg1967} a gauge theory could provide a unification of the electromagnetic and weak forces into the electroweak interaction. Experimental evidence supporting this theory through the discovery of the $W^{\pm}$ and $Z^{0}$ vector bosons was first captured by The European Organization for Nuclear Research (CERN) ~\cite{cern1974,arnison1983,banner1983}. The unification of electromagnetic and weak interactions necessitates symmetry requirements, which yields massless gauge bosons; however, massless $W$ and $Z$ were not consistent: neither with the short range of the weak interaction nor the experimental evidence. The weak force is the only interaction in nature that violates parity symmetry, charge conjugation, and time reversal, the latter responsible for neutrino oscillations. This is what makes the weak force so intriguing.

The massless nature of the photon and the confirmation of the existence of these three massive gauge bosons, i.e $W^{\pm}$ and $Z^{0}$, are evidence that the symmetry among the force-carrying particles was broken: a symmetry breaking mechanism would allow the electroweak bosons $W^{\pm}$ and $Z^{0}$ to acquire mass.

The Higgs Field theory came to the rescue of the standard model, allowing the weak bosons to acquire mass. The successful finding of the widely expected Higgs Boson ~\cite{higgsPaperCMS,higgsPaperATLAS} (the only scalar fundamental particle), and its 125 GeV, lighter than expected mass, was the experimental evidence required to support the Higgs field theory. However, the observed Higgs boson requires a high level of SM fine tuning through the manually set properties of the now seventeen elementary particles. This suggests an underlying, more fundamental structure to be discovered and a theory beyond the standard model. Additional reasons that are strongly coherent with this suggestion are the following: the wildly different masses of the three different families of particles, the lack of mass ratios predictions, open questions such as the structure of dark matter, and the inclusion of gravity. 

Some of the most promising hypotheses addressing the Electroweak Symmetry Breaking (EWSB) propose that the Higgs field emerges from a new strongly-interacting composite sector. An unavoidable consequence from these theories is the existence of Spin-1 vector resonances decaying to heavy vector bosons. Broadly speaking these resonances fall into two categories: electrically charged resonances, commonly referred as $W^{\prime}$, and neutral resonances $Z^{\prime}$. These resonances might be available at the TeV scale ~\cite{tevscale2014}, the current energy scale of the Large Hadron Collider (LHC). The question then becomes: can experimental evidence of the existence of such resonances be found? 

\section{Statement of purpose}

Currently, a diverse range of theories beyond the standard model provide explanations for the EWSB mechanism. As mentioned before, vector boson resonances are a prediction of a large number of SM extensions, including, but not limited to, the Electroweak Chiral Lagrangian model  ~\cite{echl2017}, the Little Higgs ~\cite{littlehiggs2007}, the Randall-Subdrum model ~\cite{randall1999}, the Minimal walking technicolor ~\cite{technicolor2007}, or models where the Higgs is a composite pseudo-Nambu-Goldstone boson ~\cite{composite2016}. 


In this analysis, I test a set of theoretical models based on a phenomenological lagrangian that contains Heavy Vector Triplets (HVT): the HVT framework, which generalizes a large number of models predicting Spin-1 resonances ~\cite{hvt2014}. I use two benchmarks: a weakly coupled extended gauge symmetry and a strongly coupled composite Higgs scenario. Further theoretical treatment of these models is provided by Refs. ~\cite{hvt2014,modelA1980,modelB2011}. Specifically, a search for an electrically charged resonance $W^{\prime}$ is performed. This search focuses on the $W^{\prime}$ fully leptonic decay: $W^{\prime}\rightarrow WZ \rightarrow \ell\nu \ell\ell$ ($\ell = e$ or $\mu$), which, despite its low branching ratio, offers low SM backgrounds and high sensitivity in the low mass resonance range. In this channel, the SM $Z^{0}$ boson decays to a pair of same flavoured, opposite charged leptons, and the electrically charged $W^{\pm}$ decays to a lepton and a neutrino $\nu$, which escapes undetected, and, therefore, needs to be reconstructed from the missing transverse energy in the Compact Muon Solenoid (CMS) detector. 


The analysis is based on proton-proton collision data, with a center of mass energy of $\sqrt{s}=13~TeV$ collected by the CMS collaboration during the 2016-2018 time period, corresponding to an integrated luminosity of $L=137.2~fb^{-1}$. 


\section{Methodology of the study}

Signal events for electrically charged diboson resonances in the 600 GeV to 4.5 TeV mass range under the theoretical assumptions of the HVT model are generated through Monte Carlo simulations (MC) using \verb|MadGraph5_aMC@NL| ~\cite{madgraph} interfaced with Pythia 8 ~\cite{pythia} for showering and hadronization, and a full detector response simulated with the \verb|Geant4| toolkit ~\cite{geant4}. Physics objects, such as electrons, muons and missing energy are reconstructed using the Particle-flow algorithm ~\cite{particleflow}. An equivalent procedure is performed to simulate SM processes with similar signatures in the final state: among them the standard model WZ, $t\overline{t}$, Z+Jets, triboson and single top productions. 


Initial studies will look for an enhancement of the Signal/Background ratio, based on the kinematic variables to find regions in the phase-space sensitive to the presence of the signal; these will include studies on the performance of the High Level Trigger, different options of lepton identification, isolation, and other detector-specific requirements. 


The accuracy of the MC simulations for the SM processes will be studied in dedicated control regions of the phase space where the signal is suppressed. Different corrections on the MC samples will be applied in order to account for the different rates of efficiencies in the data reconstruction, particle identification, and inefficiencies in the simulation of generated processes, as well as the contribution from multiple proton-to-proton interactions occurring per bunch crossing at the LHC accelerator. Data-driven techniques such as tag and probe ~\cite{tagandprobe} or the ABCD method ~\cite{wz7tev} may be applied in order to quantify lepton misidentification rates. 


The spectrum of the invariant mass for the diboson resonance candidates will be examined where a localized excess of events is expected if the resonance is found, or otherwise, an agreement with the rates predicted by the standard model within the statistical and systematic uncertainties will be found. A fit to a background-only and background plus signal hypothesis is performed through maximum likelihood estimation. Systematic uncertainties effects are treated as nuisance parameters of the likelihood function. 


\section{Contribution to the field and originality of the study}

The contribution to the physics field during the execution of this research program will be on three fronts: physics analysis, software and detector development. 
The physics analysis opens up the possibility of narrowing down the number of assumptions for new physics beyond the standard model. If the resonance is found the search for physics BSM will be tilted in favor of the Heavy Vector Triplet framework and the possibility to further constrain the models that it generalizes. No previous search for diboson resonances at  $\sqrt{s}=13~TeV$ in the fully leptonic channel has ever been performed at the $137.2~fb^{-1}$ luminosity. This search extends to the $13~TeV$ energy scale the work performed by the CMS collaboration on the 7 TeV ~\cite{wprime7TeV} and 8 TeV front ~\cite{wprime8TeV} and expands previous efforts on the $13~TeV$ scale done by the ATLAS collaboration ~\cite{wprimeAtlas}.  If diboson resonances are not found in the mass range studied, limits on the resonance cross section times the branching ratio for the diboson production can be provided, further constraining the phase space of future searches. Future studies may include the combination of results from different final states increasing the number of available statistics and therefore the sensitivity and precision of the search for narrow-width resonances. 


As part of the work within the collaboration on the software development front, a full migration of the CMS detector geometry is being performed ~\cite{dd4hep}. This migration will open the possibility of further improvement of the CMS detector’s response simulation, therefore, it benefits all future measurements and searches for new physics. 

Finally, on the hardware front, the development of detector components place a continuous challenge due to the unique characteristics of the high radiation environment generated by the experimental conditions of High Energy Physics experiments; this serves as part of the ongoing effort for the Phase II upgrade of the CMS Tracker subsystem ~\cite{phase2tdr} testing and development for the assembly procedure of Pixel modules, which is carried out in The Catholic University High Energy Physics laboratory. 


%\end{multicols*}

\clearpage

\bibliography{mybib}
\bibliographystyle{unsrt}
\end{document}
