\section{Introduction}

The electromagnetic, weak, strong and gravitational forces are known to be
fundamental in the structure of the universe. With the exception of gravity,
which strength is negligible at particle-physics scales, these forces
are well described by the Standard model (SM) and its predictions
have been succesfully tested in numerous experiments with remarkable
precission.

As suggested initially by Schwinger ~\cite{schwinger1957}, and later developed
by Weinberg ~\cite{weinberg1967} a gauge theory could provide an unification of
the electromagnetic and weak force into the so called electroweak interaction.
Experimental evidence supporting this theory through the discovery of the $W^{\pm}$
and $Z^{0}$ vector bosons was first captured by CERN ~\cite{cern1974,arnison1983,banner1983}.
The unification of electromagnetic and weak interactions does not come for free,
it comes with symmetry requirements leading to massless gauge bosons, but
massless $W$ and $Z$ were not consistent neither with the short range of
the weak interaction nor the experimental evidence.
Other intriguing properties of the the weak force make it special, it
is the only interaction in nature violating parity symmetry, charge conjugation,
and time reversal, responsible for the neutrino oscillations.

The massless nature of the photon, and the confirmation of the existance of
these three massive gauge boson i.e $W^{\pm}$ and $Z^{0}$ , put in evidence
that the symmetry among the force-carrying particles was broken, a symmetry
breaking mechanism would allow the electroweak bosons $W^{\pm}$ and $Z^{0}$ to acquire mass.

The succesful finding of the, widely expected, Higgs Boson ~\cite{higgsPaperCMS,higgsPaperATLAS}
(the only scalar fundamental particle), and its 125 GeV, ligther than expected mass,
requires a high level of SM fine tuning through the manually set properties
of the now seventeen elementary particles.
This suggests an underlaying, more fundamental structure to be discovered and
a theory beyond the standard model. Other
reasons that are strongly coherent with this suggestion are: the wildly
different masses of the three different families of particles, the lack of
mass ratios predictions, open questions such as the structure of dark matter,
let alone the inclusion of gravity.

Some of the most promising hypotheses addressing the Electroweak Symmetry
Breaking (EWSB), and ultimately the hierarchy problem, propose that the Higgs field
emerges from a new strongly-interacting composite sector. An unavoidale
consequence from these theories, is the existance of Spin-1 vector resonances
decaying to heavy vector bosons.  Broadly speaking these resonances
fall into  two categories: electrically
charged resonances, commonly refered as $W^{\prime}$ and
neutral resonances $Z^{\prime}$. These resonances might be available at the TeV
scale ~\cite{tevscale2014}, the current LHC energy range. The question then becomes:
can experimental evidence of the existance of such resonances be found?

Currently, a diverse range of theories
beyond the standard model provide explanations to the EWSB mechanism. As
metioned before, vector boson resonances are a prediction of a large number
of SM extensions, including, but not limited to, the Electroweak Chiral
Lagrangian model ~\cite{echl2017}, the Little Higgs
~\cite{littlehiggs2007}, the Randall-Subdrum  model ~\cite{randall1999}, the
Minimal walking technicolor ~\cite{technicolor2007}, or models where the Higgs
is a composite pseudo-Nambu-Goldstone boson ~\cite{composite2016}.

In this analysis, we test a set of theoretical models based on a phenomenological
lagrangian that contains Heavy Vector Triplets (HVT): the HVT framework, which
generalizes a large number of models predicting Spin-1 resonances ~\cite{hvt2014}.
Using two benchmarks: A weakly coupled extended gauge symmetry and a
strongly coupled composite higgs scenario. Further theoretical treatment of these models
is provided by Refs. ~\cite{hvt2014,modelA1980,modelB2011}. Specifically, a search
for an electrically charged resonance $W^{\prime}$ is performed. This search focuses
on the $W^{\prime}$ fully leptonic decay:
$W^{\prime}\rightarrow WZ \rightarrow \ell\nu \ell\ell$ ($\ell = e$ or $\mu$),
which, despite its low branching ratio, offers low SM backgrounds and high
sensitivity in the low mass resonance range. In this channel, the SM $Z^{0}$
boson decays to a pair of same flavoured, opposite charged leptons and the
electrically charged $W^{\pm}$ decays to a lepton and a neutrino $\nu$ which escapes
undetected and therefore needs to be reconstructed from the missing transverse
energy in the Compact Muon Solenoid (CMS) detector.

The analysis is based on proton-proton collision data, with a center of mass
energy of $\sqrt{s}=13$ TeV collected by the CMS collaboration during the the
2016-2018 period, corresponding to an integrated luminosity of $L=137.2~fb^{-1}$.





