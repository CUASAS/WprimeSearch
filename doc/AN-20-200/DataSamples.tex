\section{Data Samples}

The datasets used for the analysis presented in this note were collected using
the CMS detector during the period 2016-2018. Data was recorded at the
center-of-mass energy $\sqrt{s}=13~\TeV$ and a total of $137.4~\fbinv$ was collected
and reconstructed with the 10.2.18 version of the CMS software. The standard
CMS selection of good runs and luminosity sections was applied
(see Table \ref{tab:GoldenJson}).


\begin{table}[h]
\centering
\caption{List of HLT requirements and its associated dataset.}
\begin{tabular}{|l|l|l|}
\hline
Year & Dataset & HLT                \\ \hline
2016 & SingleMuon     & HLT\_TkMu50 \\
     &                & HLT\_Mu50   \\
     & SingleElectron & HLT\_Ele27\_WPTight\_Gsf  \\
     & SinglePhoton   & HLT\_Photon175            \\ \hline
2017 & SingleMuon     & HLT\_Mu50       \\
     &                & HLT\_OldMu100   \\
     &                & HLT\_TkMu100    \\
     & SingleElectron & HLT\_Ele35\_WPTight\_Gsf  \\
     & SinglePhoton   & HLT\_Photon200            \\ \hline
2018 & SingleMuon & HLT\_Mu50     \\
     &            & HLT\_OldMu100 \\
     &            & HLT\_TkMu100  \\
     & EGamma     & HLT\_Ele32\_WPTight\_Gsf \\
     &            & HLT\_Photon200           \\ \hline
\end{tabular}
\label{tab:HLTDatasets}
\end{table}

The analysis makes use of datasets which combine several High Level Trigger
(HLT) paths (Table \ref{tab:HLTDatasets}). The complete list of datasets
used are shown in Table \ref{tab:Datasets}, each
dataset contains in the order of hundreds of millions events. As an event
may trigger different HLTs it may be stored in multiple datasets simultaneously.
In order to prevent double counting an index list is built containing unique
references to events based on their run and event number. After the final selection
no overlap is found between channels.

\begin{table}[h]
\centering
\caption{List of datasets used in the analysis.}
\begin{tabular}{|l|l|l|l|}
\hline
Year & Dataset & Run & Version \\ \hline
2016 & SingleMuon     & B           & ver2-v1 \\
     &                & C,D,E,F,G,H & v1      \\
     & SingleElectron & B           & ver2-v1 \\
     &                & C,D,E,F,G,H & v1      \\
     & SinglePhoton   & B           & ver2-v1 \\
     &                & C,D,E,F,G,H & v1      \\ \hline
2017 & SingleMuon     & B,C,D,E,F & v1 \\
     & SingleElectron & B,C,D,E,F & v1 \\
     & SinglePhoton   & B,C,D,E,F & v1 \\\hline
2018 & SingleMuon & A,B,C,D & v1 \\
     & EGamma     & A,B,C,D & v1 \\ \hline
\end{tabular}
\label{tab:Datasets}
\end{table}


\begin{table}[htbp]
  \footnotesize
  \topcaption{
    The golden JSON files
  }
  \centering
  \label{tab:gJSON}
  \begin{tabular}{ c l }
    \hline
    Year & File \\
    \hline
    2016 & Cert\_271036-284044\_13TeV\_ReReco\_07Aug2017\_Collisions16\_JSON.txt \\
    2017 & Cert\_294927-306462\_13TeV\_PromptReco\_Collisions17\_JSON.txt \\
    2018 & Cert\_314472-325175\_13TeV\_PromptReco\_Collisions18\_JSON.txt \\
    \hline
  \end{tabular}
  \label{tab:GoldenJson}
\end{table}

\begin{table}
  \caption{Run 2 Luminosity}
 \begin{center}
 \begin{tabular}{cc}\hline\hline
 Year & Luminosity [$fb^{-1}$] \\ \hline\hline
 2016 & 35.92  \\
 2017 & 41.43 \\
 2018 & 59.74 \\
 \end{tabular}
 \end{center}
 \label{tab:LuminosityPerYear}
\end{table}
